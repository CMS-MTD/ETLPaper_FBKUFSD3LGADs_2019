%% 
%% Copyright 2007, 2008, 2009 Elsevier Ltd
%% 
%% This file is part of the 'Elsarticle Bundle'.
%% ---------------------------------------------
%% 
%% It may be distributed under the conditions of the LaTeX Project Public
%% License, either version 1.2 of this license or (at your option) any
%% later version.  The latest version of this license is in
%%    http://www.latex-project.org/lppl.txt
%% and version 1.2 or later is part of all distributions of LaTeX
%% version 1999/12/01 or later.
%% 
%% The list of all files belonging to the 'Elsarticle Bundle' is
%% given in the file `manifest.txt'.
%% 

%% Template article for Elsevier's document class `elsarticle'
%% with numbered style bibliographic references
%% SP 2008/03/01

\documentclass[preprint,1p]{elsarticle}
\biboptions{numbers,sort&compress}

%% Use the option review to obtain double line spacing
%% \documentclass[authoryear,preprint,review,12pt]{elsarticle}

%% Use the options 1p,twocolumn; 3p; 3p,twocolumn; 5p; or 5p,twocolumn
%% for a journal layout:
%% \documentclass[final,1p,times]{elsarticle}
%% \documentclass[final,1p,times,twocolumn]{elsarticle}
%% \documentclass[final,3p,times]{elsarticle}
%% \documentclass[final,3p,times,twocolumn]{elsarticle}
%% \documentclass[final,5p,times]{elsarticle}
%% \documentclass[final,5p,times,twocolumn]{elsarticle}

%% For including figures, graphicx.sty has been loaded in
%% elsarticle.cls. If you prefer to use the old commands
%% please give \usepackage{epsfig}

%% The amssymb package provides various useful mathematical symbols
\usepackage{amssymb}
\usepackage{lineno}
\usepackage{hyperref}
\usepackage{siunitx}
\usepackage{multirow}
\usepackage{wasysym}
%\usepackage[percent]{overpic}
%\usepackage[usenames,dvipsnames,svgnames,table]{xcolor}
%\usepackage{cleveref}

%% The amsthm package provides extended theorem environments
%% \usepackage{amsthm}

%% The lineno packages adds line numbers. Start line numbering with
%% \begin{linenumbers}, end it with \end{linenumbers}. Or switch it on
%% for the whole article with \linenumbers.
%% \usepackage{lineno}

\journal{Nucl. Instrum. Meth. A}

\begin{document}

\linenumbers

\begin{frontmatter}

%% Title, authors and addresses

%% use the tnoteref command within \title for footnotes;
%% use the tnotetext command for theassociated footnote;
%% use the fnref command within \author or \address for footnotes;
%% use the fntext command for theassociated footnote;
%% use the corref command within \author for corresponding author footnotes;
%% use the cortext command for theassociated footnote;
%% use the ead command for the email address,
%% and the form \ead[url] for the home page:
%% \title{Title\tnoteref{label1}}
%% \tnotetext[label1]{}
%% \author{Name\corref{cor1}\fnref{label2}}
%% \ead{email address}
%% \ead[url]{home page}
%% \fntext[label2]{}
%% \cortext[cor1]{}
%% \address{Address\fnref{label3}}
%% \fntext[label3]{}

\title{Studies of FBK UFSD3 low-gain avalanche detectors}

%% use optional labels to link authors explicitly to addresses:
%% \author[label1,label2]{}
%% \address[label1]{}
%% \address[label2]{}

\author[1]{A.~Apresyan\corref{cor}}\ead{apresyan@fnal.gov}
\author[5,7]{R.~Arcidiacono}
\author[5]{N.~Cartiglia}
\author[8]{M.~Carulla}
\author[2]{O.~Cerri}
\author[1]{G.~Derylo}
\author[5]{M.~Ferrero}
\author[8]{D.~Flores}
\author[9]{A.~Ghassemi}
\author[3]{H.~Al Ghoul}
\author[1]{R.~Heller}
\author[8]{S.~Hidalgo}
\author[1]{M.~Hussain}
\author[9]{S.~Kamada}
\author[1]{S.~Los}
\author[5]{M.~Mandurrino}
\author[8]{A.~Merlos}
\author[3]{N.~Minafra}
\author[4]{A.~Ocharova}
\author[8]{G.~Pellegrini}
\author[1,2]{C.~Pena}
\author[8]{D.~Quirion}
\author[3]{C.~Royon}
\author[5]{V.~Sola}
\author[2]{M.~Spiropulu}
\author[5]{A.~Staiano}
\author[1]{L.~Uplegger}
\author[2]{S.~Xie}

\address[1]{Fermi National Accelerator Laboratory, Batavia, IL, USA}
\address[2]{California Institute of Technology, Pasadena, CA, USA}
\address[3]{University of Kansas, KS, USA}
\address[4]{University of California Santa Barabara, CA, USA}
\address[5]{INFN, Torino, Italy}
\address[6]{Universit\`a di Torino, Torino, Italy}
\address[7]{Universit\`a del Piemonte Orientale, Italy}
\address[8]{Centro Nacional de Microelectr\'{o}nica (IMB-CNM-CSIC), Barcelona, Spain}
\cortext[cor]{Corresponding author}

\begin{abstract}
%% Text of abstract
abstract
\end{abstract}

\begin{keyword}
%% keywords here, in the form: keyword \sep keyword

%% PACS codes here, in the form: \PACS code \sep code
Silicon \sep Timing \sep LGAD \sep Test Beam
%% MSC codes here, in the form: \MSC code \sep code
%% or \MSC[2008] code \sep code (2000 is the default)

\end{keyword}

\end{frontmatter}

\tableofcontents

%% \linenumbers

%% main text
\section{Introduction} 


The paper is organized as follows: the experimental setup and data analysis
procedures are described in Sec.~\ref{sec:setup}; the tested LGAD sensors and
their operating conditions are discussed in Sec.~\ref{sec:sensors}; 
studies of sensor irradiation tolerance are presented in 
Sec.~\ref{sec:irradiationStudies}; studies of large area sensors are presented 
in Sec.~\ref{sec:largeAreaStudies}; and the conclusion is given in Sec.~\ref{sec:conclusion}.

\section{Experimental Setup and Data Analysis} 
\label{sec:setup}

%FTBF facility
%Readout electronics%
%Brief description of analysis selection and reconstruction algorithms



\section{LGAD Sensors}
\label{sec:sensors}

% Describe UFSD3 and UFSD2 sensor productions
% What's new with respect to previous productions? 
% expected features?



\section{Sensor radiation tolerance}
\label{sec:irradiationStudies}

% UFSD3 W5 - small sensors (1x3mm^2)
%   amplitude & time resolution vs irradiation
%   amplitude & time resolution vs BV
%   amplitude/time mean/time resolution vs position uniformity - 2D and 1D

\section{Uniformity of large area sensors}
\label{sec:largeAreaStudies}

% UFSD2 2x8 16-CH sensor - 
%   amplitude (& efficiency if we can get near 100%) vs position
%   time mean vs position - if we can take more data of this with Photek
%   Measurement of pixel gaps?

 
\section{Conclusion}
\label{sec:conclusion} 


\section*{Acknowledgment}

We thank the FTBF personnel and Fermilab accelerator's team for very good beam
conditions during our test beam time. We also appreciate the technical support
of the Fermilab SiDet department for the rapid production of wire-bonded and
packaged LGAD assemblies. We would like to thank Alan Prosser and Ryan Rivera
for their critical help in setting up the DAQ and trigger chain. We thank Ned
Spencer, Max Wilder, and Forest McKinney-Martinez for their technical
assistance, and the CNM and HPK manufacturing team. We acknowledge the help of
V. Cindro and I. Mandic with the neutron irradiations. 

This document was prepared using the resources of the Fermi National Accelerator
Laboratory (Fermilab), a U.S. Department of Energy, Office of Science, HEP User
Facility. Fermilab is managed by Fermi Research Alliance, LLC (FRA), acting
under Contract No. DE-AC02-07CH11359. Part of this work was performed within the
framework of the CERN RD50 collaboration.

This work was supported by the Fermilab LDRD 2017.027; by the United States
Department of Energy grant DE-FG02-04ER41286; by the California Institute of
Technology High Energy Physics under Contract DE-SC0011925; by the European
Union's Horizon 2020 Research and Innovation funding program, under Grant
Agreement no. 654168 (AIDA-2020) and Grant Agreement no. 669529 (ERC
UFSD669529); by the Italian Ministero degli Affari Esteri and INFN Gruppo V; and
by the Spanish Ministry of Economy, Industry and Competitiveness through the
Particle Physics National Program (ref. FPA2014-55295-C3-2-R and
FPA2015-69260-C3-3-R) co-financed with FEDER funds.


% The Appendices part is started with the command \appendix;
% appendix sections are then done as normal sections

%\appendix
%\section{Appendix A}



% \section{}
% \label{}

%% If you have bibdatabase file and want bibtex to generate the
%% bibitems, please use
%%
%%  \bibliographystyle{elsarticle-num} 
%%  \bibliography{<your bibdatabase>}

%% else use the following coding to input the bibitems directly in the
%% TeX file.

\bibliography{LGAD_May2017_FNALTB}{}
\bibliographystyle{ieeetr} 

%\begin{thebibliography}{00}

%% \bibitem{label}
%% Text of bibliographic item

%\bibitem{}

%\end{thebibliography}
\end{document}
\endinput
%%
%% End of file `elsarticle-template-num.tex'.


























